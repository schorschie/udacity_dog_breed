\documentclass[paper=A4, DIV=calc, parskip=half]{scrartcl}

% Udacity Links
% -------------

% Project Review Rubric:
% https://review.udacity.com/#!/rubrics/2345/view

% Examples:
% https://github.com/udacity/machine-learning/blob/master/projects/capstone/report-example-1.pdf
% https://github.com/udacity/machine-learning/blob/master/projects/capstone/report-example-3.pdf

% Link to Capstone Proposal Review
% https://review.udacity.com/#!/reviews/2779272

% Packages
% --------

\RequirePackage{hyperref}
\RequirePackage{xcolor}
\hypersetup{%
  colorlinks=false,% hyperlinks will be black
  linkbordercolor=darkgray,% hyperlink borders will be red
  pdfborderstyle={/S/U/W 1}% border style will be underline of width 1pt
}
\RequirePackage{graphicx}
\RequirePackage[english]{babel}

% Document Head
% -------------

\title{Dog Breed Identification}
\subtitle{Project Report}
\author{Grzegorz Lippe}
\date{\today}

\begin{document}

\maketitle

\section*{Definition}

\subsection*{Project Overview}

% Student provides a high-level overview of the project in layman’s terms. Background
% information such as the problem domain, the project origin, and related data sets or
% input data is given. 

\subsection*{Problem Statement}

% The problem which needs to be solved is clearly defined. A strategy for solving the
% problem, including discussion of the expected solution, has been made.

\subsection*{Metrics}

% Metrics used to measure the performance of a model or result are clearly defined.
% Metrics are justified based on the characteristics of the problem.

\section*{Analysis}

\subsection*{Data Exploration}

% If a dataset is present, features and calculated statistics relevant to the problem have
% been reported and discussed, along with a sampling of the data. In lieu of a dataset, a
% thorough description of the input space or input data has been made. Abnormalities or
% characteristics of the data or input that need to be addressed have been identified.

\subsection*{Exploratory Visualization}

% A visualization has been provided that summarizes or extracts a relevant characteristic
% or feature about the dataset or input data with thorough discussion. Visual cues are
% clearly defined.

\subsection*{Algorithms and Techniques}

% Algorithms and techniques used in the project are thoroughly discussed and properly
% justified based on the characteristics of the problem.

\subsection*{Benchmark}

% Student clearly defines a benchmark result or threshold for comparing performances of
% solutions obtained.

\section*{Methodology}

\subsection*{Data Preprocessing}

% All preprocessing steps have been clearly documented. Abnormalities or characteristics
% of the data or input that needed to be addressed have been corrected. If no data
% preprocessing is necessary, it has been clearly justified.

\subsection*{Implementation}

% The process for which metrics, algorithms, and techniques were implemented with the
% given datasets or input data has been thoroughly documented. Complications that occurred
% during the coding process are discussed.

\subsection*{Refinement}

% The process of improving upon the algorithms and techniques used is clearly documented.
% Both the initial and final solutions are reported, along with intermediate solutions, if
% necessary.

\section*{Results}

\subsection*{Model Evaluation and Validation}

% The final model’s qualities—such as parameters—are evaluated in detail. Some type of
% analysis is used to validate the robustness of the model’s solution.

\subsection*{Justification}

% The final results are compared to the benchmark result or threshold with some type of
% statistical analysis. Justification is made as to whether the final model and solution
% is significant enough to have adequately solved the problem.

\end{document}